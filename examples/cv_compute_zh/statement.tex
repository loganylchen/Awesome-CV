%-------------------------------------------------------------------------------
%	SECTION TITLE
%-------------------------------------------------------------------------------
\cvsection{Personal Value Statement}


%-------------------------------------------------------------------------------
%	CONTENT
%-------------------------------------------------------------------------------
\begin{cvparagraph}

%---------------------------------------------------------
在生物信息领域,兼具深度产业实践经验与系统学术训练的人才极为稀缺,而我恰好形成了“4年业界实战+博士学术深耕”的独特竞争力。

4年期间,我从生物信息工程师逐步成长为生物信息总监,完整经历技术落地、团队管理、项目统筹全流程:在深圳安吉康尔科技有限公司担任总监时,牵头构建遗传病检测全自动检测平台,主导从技术方案设计到落地应用的全链条工作,同时负责变异检测一体机开发,推动技术向产品转化;在北京橡鑫生物科技有限公司担任项目组长期间,带领团队研发的血液肿瘤FLT3-ITD NGS检测方法,实现定量结果与金标准毛细管电泳稳定一致,直接支撑临床应用。这些经历让我具备了极强的项目落地能力、团队管理能力与技术转化思维,能够快速识别产业需求、解决实际问题。

后续在香港中文大学的博士阶段,我围绕生物信息学算法开发与计算生物学研究深耕,参与开发BALEEN、FIN、TAPE等多个核心科研项目,其中TAPE项目成果已发表于Nature Communications(IF:15.7),并为2022年香港RGC GRF获批项目、2024年1+1+1联合基金获批项目提供关键技术支撑。博士阶段的训练进一步夯实了我的科研创新能力,让我能够将业界实践中发现的问题转化为学术研究方向,再通过科研成果反哺产业发展,形成“实践-研究-转化”的闭环。

\subsection{适配海南高层次人才政策的核心价值}
\begin{cvitems} % Description(s)
    \item {契合“四方之才”“南海”人才计划的产业与科研需求: 海南“四方之才”汇聚计划强调领军人才的项目推动能力,“南海”人才开发计划注重创新、创业人才的实践转化能力,而我的经历恰好与之高度匹配。一方面,凭借4年业界管理经验,我熟悉生物信息检测平台搭建、临床检测项目落地全流程,能够快速对接海南医疗健康、基因技术等产业需求,推动相关技术在本地的产业化应用,符合领军人才、创业人才的选拔导向;另一方面,博士阶段的科研积累让我在算法开发、高通量数据分析等领域具备创新能力,可依托海南的科研资源开展前沿研究,助力“南海创新人才”相关项目落地,为海南生物信息学科发展注入新动能。}
    \item {支撑海南医疗健康产业发展的实践能力: 海南正大力发展医疗健康、基因检测等产业,而我在血液肿瘤检测、遗传病检测等领域拥有丰富的技术开发与项目落地经验:曾开发全自动监控及运行平台保障血液肿瘤临床商业检测无错运行,参与血液肿瘤IVD研发,拥有5项生物信息检测相关发明专利(如“一种检测拷贝数变异的方法、装置和存储介质”“一种用于检测血液病相关体细胞突变的装置”等),能够直接为海南本地医疗机构、基因技术企业提供技术支持,推动临床检测技术升级,助力海南医疗健康产业高质量发展。}
    \item {满足高层次人才团队建设的管理潜力: 海南高层次人才引进政策中,无论是“四方之才”的优秀人才团队资助,还是海南师范大学对科研团队保障的要求,均强调人才的团队组建与管理能力。我曾带领团队完成遗传病检测平台搭建、血液肿瘤检测方法研发等核心项目,具备组建高效科研与技术团队的经验,能够根据海南本地需求整合资源,招聘、培养青年科研人员,推动团队快速形成战斗力,适配政策中对团队建设的支持导向,为海南生物信息领域人才梯队建设贡献力量。
}
    \item {(基金)主要贡献: 2022年香港RGC GRF项目(获批); 2024年香港RGC GRF项目(获批); 2024年1+1+1联合基金项目(获批)}
\end{cvitems}


\subsection{未来在琼发展规划:以“实战+科研”助力海南产业与学术双提升}
若成功入选海南高层次人才,我将充分发挥自身“实战+科研”的优势:在产业层面,对接海南医疗健康企业、医疗机构需求,推动生物信息检测技术的本地化落地,助力海南打造基因检测、精准医疗产业集群;在科研层面,依托海南师范大学等平台,组建生物信息科研团队,围绕热带疾病检测、海洋生物基因分析等海南特色领域开展研究,争取获批“南海”人才相关项目,提升海南在生物信息领域的学术影响力;同时,积极参与人才培养,将业界实践经验融入教学,为海南培养兼具技术能力与实践思维的生物信息人才,实现个人发展与海南建设的深度绑定,为海南自由贸易港建设贡献专业力量。

\end{cvparagraph}