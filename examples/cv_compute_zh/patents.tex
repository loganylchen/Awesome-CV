%-------------------------------------------------------------------------------
%	SECTION TITLE
%-------------------------------------------------------------------------------
\cvsection{Patents}


%-------------------------------------------------------------------------------
%	CONTENT
%-------------------------------------------------------------------------------


\begin{table}[h]

\centering
\tablefont
\begin{tabularx}{\textwidth}{>{\raggedright\arraybackslash}X| X| X| c}
\toprule
{\textbf{专利名称}} & {\textbf{申请号}} & {\textbf{发明人}} & {\textbf{专利类型}} \\
\midrule

一种检测拷贝数变异的方法、装置和存储介质 & CN202010184960.3 & \textbf{\textcolor{red}{陈玥茏}}; 刘永初; 李阳; 刘阳; 吕佩涛 & 发明专利 \\
\addlinespace



用于检测遗传性疾病基因变异的方法、装置及终端设备 & CN201811021290.2 & \textbf{\textcolor{red}{陈玥茏}}; 刘永初; 刘阳; 李阳; 吕佩涛 & 发明专利\\
\addlinespace



一种用于检测血液病相关体细胞突变的装置 & CN201710067161.6 & \textbf{\textcolor{red}{陈玥茏}}; 侯光远; 李停; 方真; 刘伟; 玄兆伶; 李大为; 梁峻彬; 陈重建 & 发明专利\\
\addlinespace

一种用于利用循环肿瘤DNA样本检测体细胞突变的装置 & CN201710067084.4 & \textbf{\textcolor{red}{陈玥茏}}; 侯光远; 蔡丽丽; 李雪峰; 方真; 玄兆伶; 李大为; 梁峻彬; 陈重建 &发明专利 \\
\addlinespace

一种用于利用肿瘤FFPE样本检测体细胞突变的装置 & CN201710067031.2 & \textbf{\textcolor{red}{陈玥茏}}; 侯光远; 刘卉; 陈玉洁; 王旺; 玄兆伶; 李大为; 梁峻彬; 陈重建 &发明专利 \\
\bottomrule
\end{tabularx}
\end{table}

