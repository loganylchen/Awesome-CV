%-------------------------------------------------------------------------------
%	SECTION TITLE
%-------------------------------------------------------------------------------
\cvsection{Summary}


%-------------------------------------------------------------------------------
%	CONTENT
%-------------------------------------------------------------------------------
\begin{cvparagraph}

%---------------------------------------------------------
香港中文大学生物信息学博士,\textcolor{red}{专注于生物信息学算法开发与计算生物学研究,擅长通过算法开发、高通量数据分析及多组学计算流程构建解决复杂生物学问题。}。

\begin{cvitems} % Description(s)
    \item {BALEEN: 利用dynamic time warping(DTW)算法及贝叶斯模型识别Nanopore直接RNA测序数据上的mRNA分子上的修饰位点。(撰写paper中,目标: Nature Communications)}
    \item {FIN: 基于transfomer模型,构建新的gene\-fusion检测模型,达到高准确及高敏感的fusion检测。(开发中,目标: Nature Communications)}
    \item {TAPE: 借助深度自动编码器架构与自适应训练方案,整合批量 RNA-seq 和单细胞 RNA-seq 数据,实现组织自适应的反卷积及细胞类型特异性基因表达预测,具备高精度、高稳健性与快速分析能力。(已发表)}
    \item {(基金)主要贡献: 2022年香港RGC GRF项目(获批); 2024年香港RGC GRF项目(在审); 2024年1+1+1联合基金项目(获批)}
\end{cvitems}


\end{cvparagraph}