%-------------------------------------------------------------------------------
%	SECTION TITLE
%-------------------------------------------------------------------------------
\cvsection{Summary}


%-------------------------------------------------------------------------------
%	CONTENT
%-------------------------------------------------------------------------------
\begin{cvparagraph}

%---------------------------------------------------------
香港中文大学生物信息学博士,\textcolor {red}{兼具 4 年生物信息领域业界核心管理经验与顶尖学术科研积淀 —— 曾任生物信息总监,带领团队完成多项临床检测平台搭建、核心技术产品化落地任务,积累了从技术攻坚、团队统筹到产业转化的全链条实战能力;后深耕博士阶段研究,实现产业实践痛点与学术创新方向的精准对接,形成 “实战落地 + 科研突破” 的稀缺双重竞争力,可高效打通 “科研 - 产业” 转化闭环,推动技术成果快速落地应用、产业需求转化为学术创新课题。}
% \begin{cvitems} % Description(s)
%     \item {BALEEN: 利用dynamic time warping(DTW)算法及贝叶斯模型识别Nanopore直接RNA测序数据上的mRNA分子上的修饰位点。(最终定稿中,目标: Nature Biotechnology}
%     \item {FIN: 基于transfomer模型,构建新的gene\-fusion检测模型,达到高准确及高敏感的fusion检测。(开发中,目标: Nature Communications)}
%     \item {TAPE: 借助深度自动编码器架构与自适应训练方案,整合批量 RNA-seq 和单细胞 RNA-seq 数据,实现组织自适应的反卷积及细胞类型特异性基因表达预测,具备高精度、高稳健性与快速分析能力。(已发表)}
%     \item {(基金)主要贡献: 2022年香港RGC GRF项目(获批); 2024年香港RGC GRF项目(获批); 2024年1+1+1联合基金项目(获批)}
% \end{cvitems}


在生物信息领域,兼具深度产业实践经验与系统学术训练的人才极为稀缺,而我恰好形成了 “4 年业界实战 + 博士学术深耕” 的独特竞争力。
4 年期间,我从生物信息工程师逐步成长为生物信息总监,完整经历技术落地、团队管理、项目统筹全流程:在深圳安吉康尔科技有限公司担任总监时,牵头构建遗传病检测全自动检测平台,主导从技术方案设计到落地应用的全链条工作,同时负责变异检测一体机开发,推动技术向产品转化;在北京橡鑫生物科技有限公司担任项目组长期间,带领团队研发的血液肿瘤 FLT3-ITD NGS 检测方法,实现定量结果与金标准毛细管电泳稳定一致,直接支撑临床应用。这些经历让我具备了极强的项目落地能力、团队管理能力与技术转化思维,能够快速识别产业需求、解决实际问题。
后续在香港中文大学的博士阶段,我围绕生物信息学算法开发与计算生物学研究深耕,参与开发 BALEEN、FIN、TAPE 等多个核心科研项目,其中 TAPE 项目成果已发表于 Nature Communications(IF:15.7),并为 2022 年香港 RGC GRF 获批项目、2024 年 1+1+1 联合基金获批项目提供关键技术支撑。博士阶段的训练进一步夯实了我的科研创新能力,让我能够将业界实践中发现的问题转化为学术研究方向,再通过科研成果反哺产业发展,形成 “实践 - 研究 - 转化” 的闭环。

\end{cvparagraph}