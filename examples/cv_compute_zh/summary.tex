%-------------------------------------------------------------------------------
%	SECTION TITLE
%-------------------------------------------------------------------------------
\cvsection{Summary}


%-------------------------------------------------------------------------------
%	CONTENT
%-------------------------------------------------------------------------------
\begin{cvparagraph}

%---------------------------------------------------------
香港中文大学生物信息学博士,\textcolor{red}{拥有 4 年生物信息领域业界实战经验,曾任生物信息总监,带领团队完成生产与开发核心任务,具备丰富的团队管理、项目落地及技术转化能力。后续深耕香港中文大学生物信息学博士研究,实现业界实践经验与学术科研能力的深度融合,形成 “实战 + 科研” 的双重核心竞争力,可高效推动科研项目产业化、产业需求学术化。}

\begin{cvitems} % Description(s)
    \item {BALEEN: 利用dynamic time warping(DTW)算法及贝叶斯模型识别Nanopore直接RNA测序数据上的mRNA分子上的修饰位点。(最终定稿中,目标: Nature Biotechnology}
    \item {FIN: 基于transfomer模型,构建新的gene\-fusion检测模型,达到高准确及高敏感的fusion检测。(开发中,目标: Nature Communications)}
    \item {TAPE: 借助深度自动编码器架构与自适应训练方案,整合批量 RNA-seq 和单细胞 RNA-seq 数据,实现组织自适应的反卷积及细胞类型特异性基因表达预测,具备高精度、高稳健性与快速分析能力。(已发表)}
    \item {(基金)主要贡献: 2022年香港RGC GRF项目(获批); 2024年香港RGC GRF项目(获批); 2024年1+1+1联合基金项目(获批)}
\end{cvitems}


\end{cvparagraph}