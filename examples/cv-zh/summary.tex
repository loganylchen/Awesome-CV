%-------------------------------------------------------------------------------
%	SECTION TITLE
%-------------------------------------------------------------------------------
\cvsection{Summary}


%-------------------------------------------------------------------------------
%	CONTENT
%-------------------------------------------------------------------------------
\begin{cvparagraph}

%---------------------------------------------------------
香港中文大学生物信息学博士,\textcolor {red}{兼具 4 年生物信息领域业界核心管理经验与顶尖学术科研积淀 —— 曾任生物信息总监,带领团队完成多项临床检测平台搭建、核心技术产品化落地任务,积累了从技术攻坚、团队统筹到产业转化的全链条实战能力;
后深耕博士阶段研究,实现产业实践痛点与学术创新方向的精准对接,形成 “实战落地 + 科研突破” 的稀缺双重竞争力,可高效打通 “科研 - 产业” 转化闭环,推动技术成果快速落地应用、产业需求转化为学术创新课题。}
\\
\\
\\
\begin{cvitems} % Description(s)
    \item {BALEEN: 利用dynamic time warping(DTW)算法及贝叶斯模型识别Nanopore直接RNA测序数据上的mRNA分子上的修饰位点。(最终定稿中,目标: Nature Biotechnology}
    \item {FIN: 基于transfomer模型,构建新的gene-fusion检测模型,达到高准确及高敏感的fusion检测。(开发中,目标: Nature Communications)}
    \item {TAPE: 借助深度自动编码器架构与自适应训练方案,整合批量 RNA-seq 和单细胞 RNA-seq 数据,实现组织自适应的反卷积及细胞类型特异性基因表达预测,具备高精度、高稳健性与快速分析能力。(已发表)}\\
    \item {(基金)主要贡献: 2022年香港RGC GRF项目(获批); 2024年香港RGC GRF项目(获批); 2024年1+1+1联合基金项目(获批)}
\end{cvitems}
\\
\end{cvparagraph}
