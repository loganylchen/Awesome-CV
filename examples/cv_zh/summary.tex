%-------------------------------------------------------------------------------
%	SECTION TITLE
%-------------------------------------------------------------------------------
\cvsection{Summary}


%-------------------------------------------------------------------------------
%	CONTENT
%-------------------------------------------------------------------------------
\begin{cvparagraph}

%---------------------------------------------------------
香港中文大学生物信息学博士研究生,\textcolor{red}{专注于转录组学分析、Nanopore DRS 算法开发及疾病机制研究,整合生物信息学与实验技术解决复杂生物学问题。}

\begin{cvitems} % Description(s)
    \item {转录组及转录后调控:开发基于 Nanopore 直接 RNA 测序(DRS)的新型 RNA 修饰检测方法。构建多组学分析流程,解析转录组动态变化与疾病发生发展的关联机制。}
    \item {生物信息学算法开发:主导开发 Nanopore DRS 数据处理工具链,包括转录本组装, 定量及修饰位点识别模块。 }
    \item {代谢疾病与肿瘤机制研究: 聚焦糖尿病对肾小管 / 肾小球的差异化损伤机制,通过单细胞转录组揭示细胞异质性;探索肿瘤的转录调控网络,开发基于转录组/表观转录组特征的预后标志物。}
    \item {\textcolor{red}{(基金)主要贡献: 2022年香港RGC GRF项目(获批); 2024年香港RGC GRF项目(在审); 2024年1+1+1联合基金项目(获批)}}
\end{cvitems}


\end{cvparagraph}